\documentclass[12pt]{article}

\begin{document}\frenchspacing\raggedbottom\sloppy\sloppypar

\subsection*{Expected Future Discounted Information Gain:\\
An objective for making real-time observing decisions}

\noindent
\textbf{David W. Hogg} and \textbf{Megan Bedell}\\
\textsl{Flatiron Institute, a division of the Simons Foundation, New York City}

\paragraph{Abstract:}
The next generation of extreme-precision radial-velocity experiments to find or
confirm extra-solar planet discoveries will have to use some kind of algorithmic
decision-making if they want to maixmize their statistical power and legacy value.
There has been work on algorithmic scheduling of observations,
based on Bayesian decision theory.
Most of this work has focused on some kind of expected information gain,
and thought in terms of building a full-survey schedule of
observations.
Because of weather, time-domain interrupts, and other
unpredictable factors, observation planning is a real-time activity at the telescope,
with final target decisions made only hours or minutes before the shutter opens;
this motivates thinking about real-time decisions (rather than full-survey planning).
Discoveries made sooner are more valuable than discoveries
made later; that is, scientists have an effective discount rate on the information
they seek; this motivates thinking about discounted objectives.
And, previously unobserved stars don't have known values for systemic velocity,
variability, or activity, so the first few observations made of a star don't themselves
bring much information about exoplanet companions; that is, the point of an observation
is not just to learn, but to make opportunities for future learning.
All these considerations lead us to develop an objective for making real-time observing
decisions that involves the expectation value of future-discounted information gain (EFDIG).
We develop, analyze, and criticize the EFDIG, and propose practical methods for computing
it in real-time applications.
We concentrate on two simple cases:
A \textsl{Terra-Hunting-Experiment}-like setting in which
dozens of stars will be monitored for a decade,
and A \textsl{TESS}-follow-up-like setting in which
individually observed single transits need to be confirmed as coming from long-period planets,
but the objective and the analysis is general.

\section{Introduction}

Talk to me.

\end{document}
